\documentclass[12pt]{article}
\usepackage{pmmeta}
\pmcanonicalname{StephenWilliamHawking}
\pmcreated{2013-03-22 18:44:02}
\pmmodified{2013-03-22 18:44:02}
\pmowner{bci1}{20947}
\pmmodifier{bci1}{20947}
\pmtitle{Stephen William Hawking}
\pmrecord{26}{41503}
\pmprivacy{1}
\pmauthor{bci1}{20947}
\pmtype{Biography}
\pmcomment{trigger rebuild}
\pmclassification{msc}{85-00}
\pmclassification{msc}{83-02}
\pmclassification{msc}{83-01}
\pmsynonym{Stephen Hawking}{StephenWilliamHawking}
%\pmkeywords{big-bang theory}
%\pmkeywords{Hawking radiation}
%\pmkeywords{black hole evaporation and entropy}
%\pmkeywords{GalileoGalilei}
\pmrelated{BiographiesOnPlanetMath}

\endmetadata

% this is the default PlanetMath preamble. as your knowledge
% of TeX increases, you will probably want to edit this, but
\usepackage{amsmath, amssymb, amsfonts, amsthm, amscd, latexsym}
%%\usepackage{xypic}
\usepackage[mathscr]{eucal}
% define commands here
\theoremstyle{plain}
\newtheorem{lemma}{Lemma}[section]
\newtheorem{proposition}{Proposition}[section]
\newtheorem{theorem}{Theorem}[section]
\newtheorem{corollary}{Corollary}[section]
\theoremstyle{definition}
\newtheorem{definition}{Definition}[section]
\newtheorem{example}{Example}[section]
%\theoremstyle{remark}
\newtheorem{remark}{Remark}[section]
\newtheorem*{notation}{Notation}
\newtheorem*{claim}{Claim}
\renewcommand{\thefootnote}{\ensuremath{\fnsymbol{footnote%%@
}}}
\numberwithin{equation}{section}
\newcommand{\Ad}{{\rm Ad}}
\newcommand{\Aut}{{\rm Aut}}
\newcommand{\Cl}{{\rm Cl}}
\newcommand{\Co}{{\rm Co}}
\newcommand{\DES}{{\rm DES}}
\newcommand{\Diff}{{\rm Diff}}
\newcommand{\Dom}{{\rm Dom}}
\newcommand{\Hol}{{\rm Hol}}
\newcommand{\Mon}{{\rm Mon}}
\newcommand{\Hom}{{\rm Hom}}
\newcommand{\Ker}{{\rm Ker}}
\newcommand{\Ind}{{\rm Ind}}
\newcommand{\IM}{{\rm Im}}
\newcommand{\Is}{{\rm Is}}
\newcommand{\ID}{{\rm id}}
\newcommand{\GL}{{\rm GL}}
\newcommand{\Iso}{{\rm Iso}}
\newcommand{\Sem}{{\rm Sem}}
\newcommand{\St}{{\rm St}}
\newcommand{\Sym}{{\rm Sym}}
\newcommand{\SU}{{\rm SU}}
\newcommand{\Tor}{{\rm Tor}}
\newcommand{\U}{{\rm U}}
\newcommand{\A}{\mathcal A}
\newcommand{\Ce}{\mathcal C}
\newcommand{\D}{\mathcal D}
\newcommand{\E}{\mathcal E}
\newcommand{\F}{\mathcal F}
\newcommand{\G}{\mathcal G}
\newcommand{\Q}{\mathcal Q}
\newcommand{\R}{\mathcal R}
\newcommand{\cS}{\mathcal S}
\newcommand{\cU}{\mathcal U}
\newcommand{\W}{\mathcal W}
\newcommand{\bA}{\mathbb{A}}
\newcommand{\bB}{\mathbb{B}}
\newcommand{\bC}{\mathbb{C}}
\newcommand{\bD}{\mathbb{D}}
\newcommand{\bE}{\mathbb{E}}
\newcommand{\bF}{\mathbb{F}}
\newcommand{\bG}{\mathbb{G}}
\newcommand{\bK}{\mathbb{K}}
\newcommand{\bM}{\mathbb{M}}
\newcommand{\bN}{\mathbb{N}}
\newcommand{\bO}{\mathbb{O}}
\newcommand{\bP}{\mathbb{P}}
\newcommand{\bR}{\mathbb{R}}
\newcommand{\bV}{\mathbb{V}}
\newcommand{\bZ}{\mathbb{Z}}
\newcommand{\bfE}{\mathbf{E}}
\newcommand{\bfX}{\mathbf{X}}
\newcommand{\bfY}{\mathbf{Y}}
\newcommand{\bfZ}{\mathbf{Z}}
\renewcommand{\O}{\Omega}
\renewcommand{\o}{\omega}
\newcommand{\vp}{\varphi}
\newcommand{\vep}{\varepsilon}
\newcommand{\diag}{{\rm diag}}
\newcommand{\grp}{{\mathbb G}}
\newcommand{\dgrp}{{\mathbb D}}
\newcommand{\desp}{{\mathbb D^{\rm{es}}}}
\newcommand{\Geod}{{\rm Geod}}
\newcommand{\geod}{{\rm geod}}
\newcommand{\hgr}{{\mathbb H}}
\newcommand{\mgr}{{\mathbb M}}
\newcommand{\ob}{{\rm Ob}}
\newcommand{\obg}{{\rm Ob(\mathbb G)}}
\newcommand{\obgp}{{\rm Ob(\mathbb G')}}
\newcommand{\obh}{{\rm Ob(\mathbb H)}}
\newcommand{\Osmooth}{{\Omega^{\infty}(X,*)}}
\newcommand{\ghomotop}{{\rho_2^{\square}}}
\newcommand{\gcalp}{{\mathbb G(\mathcal P)}}
\newcommand{\rf}{{R_{\mathcal F}}}
\newcommand{\glob}{{\rm glob}}
\newcommand{\loc}{{\rm loc}}
\newcommand{\TOP}{{\rm TOP}}
\newcommand{\wti}{\widetilde}
\newcommand{\what}{\widehat}
\renewcommand{\a}{\alpha}
\newcommand{\be}{\beta}
\newcommand{\ga}{\gamma}
\newcommand{\Ga}{\Gamma}
\newcommand{\de}{\delta}
\newcommand{\del}{\partial}
\newcommand{\ka}{\kappa}
\newcommand{\si}{\sigma}
\newcommand{\ta}{\tau}
\newcommand{\lra}{{\longrightarrow}}
\newcommand{\ra}{{\rightarrow}}
\newcommand{\rat}{{\rightarrowtail}}
\newcommand{\oset}[1]{\overset {#1}{\ra}}
\newcommand{\osetl}[1]{\overset {#1}{\lra}}
\newcommand{\hr}{{\hookrightarrow}}
\begin{document}
\subsection{Stephen William Hawking, PhD, CH, CBE, FRS, FRSA, Lucasian Professor of Mathematics at the University of Cambridge, UK (rumored that he `intends to retire from this post in 2009')} 
 Prominent, British mathematical physicist specialized in cosmological and big-bang theories, was born on January 8th, 1942 (exactly 300 years after the death of the famous Italian physicist -and recognized Founder of Experimental Physics- Galileo Galilei) in Oxford, England.  Also a Fellow of Gonville and Caius College, Cambridge, UK, and distinguished research chair at Waterloo's Perimeter Institute for Theoretical Physics.

The following is his biographical, verbatim statement at his official website:
"Stephen Hawking has worked on the basic laws which govern the universe. With Roger Penrose he showed that Einstein's General Theory of Relativity implied space and time would have a beginning in the Big Bang and an end in black holes. These results indicated that it was necessary to unify General Relativity with Quantum Theory, the other great Scientific development of the first half of the 20th Century. One consequence of such a unification that he discovered was that black holes should not be completely black, but rather should emit radiation and eventually evaporate and disappear. Another conjecture is that the universe has no edge or boundary in imaginary time. This would imply that the way the universe began was completely determined by the laws of science.''

 His many publications include: ``{\em The Large Scale Structure of Spacetime}'' with G. F. R. Ellis, ``{\em General Relativity: An Einstein Centenary Survey.}'', with. W Israel, and ``{\em 300 Years of Gravity.}'', with W. Israel. Stephen Hawking has several popular books published; one of his best sellers is: ``{\em A Brief History of Time, Black Holes and Baby Universes and Other Essays.}''. 

 Hawking's proved-- together with Sir Roger Penrose-- theorems regarding singularities in the framework of general relativity; he also made the apparently paradoxical theoretical prediction that ``black holes should emit radiation'', which is currently known as the `Hawking radiation' (or as `the Bekenstein-Hawking radiation' because the former
predicted the same effect on a thermodynamic, entropic basis, at first denied by Hawking, but later accepted as valid). Hawking's scientific and popular science contributions cover over 40 years; both his books and public appearances have placed him in the limelight as one of the most prominent academic celebrity and world-renowned theoretical physicist (but no Swedish prize award so far ?!).

 Professor Stephen William Hawking has at least twelve honorary degrees. He was awarded the CBE in 1982, and was made a Companion of Honour in 1989. He is a Fellow of The Royal Society, a Member of the US National Academy of Sciences,
as well as the recipient of many awards, medals and prizes, including a recognition from his most catholic
sanctity- the Pope of Rome- for his theory on the `creation' of the Universe at/by the Big-Bang, and also for not attempting to answer the question ``what was there before the Big-bang?''. He thus became selected as a lifetime Member of the Pontifical Academy of Science in the Vatican (also known as the Academy ``dei Lincei''; the leader of the first version of this Italian Academy was the famous physicist and astronomer Galileo Galilei, but it was dissolved by the Roman Catholic Church immediately after the death of its founder, only to be re-instated by Pope Pius IX in 1847 as ``Accademia Pontificia dei Nuovi Lincei'' (i.e., ''the Pontifical Academy of the New Lynxes'')).


\PMlinkexternal{`Complete' PUBLICATIONS LIST of Professor Stephen W. Hawking, as of 28th November 2008}{http://hawking.org.uk/old-site/pdf/pub.pdf}
%%%%%
%%%%%
\end{document}
